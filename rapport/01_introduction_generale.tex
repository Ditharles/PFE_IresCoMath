\chapter*{ Introduction générale}
\addcontentsline{toc}{chapter}{ Introduction générale}

Les laboratoires universitaires sont confrontés à une diversité de besoins administratifs et matériels nécessaires à la conduite de leurs activités scientifiques. Ces besoins donnent lieu à des demandes telles que l’achat de matériel, le prêt d’équipements ou encore la participation à des conférences.

Cependant, la gestion de ces demandes est souvent manuelle, lente et peu structurée, ce qui impacte la productivité des acteurs concernés. Dans ce cadre, nous proposons la conception et la réalisation d’une application web permettant de centraliser et d’automatiser la gestion des demandes, tout en assurant un suivi rigoureux et une validation hiérarchique efficace.

Ce projet s’inscrit dans le cadre de notre formation en Informatique, spécialité Génie Logiciel, à la Faculté des Sciences de Gabès. Il a été réalisé au sein du laboratoire IreSCoMath.

Ce rapport est structuré comme suit :

\begin{itemize}
  \item Le \textbf{chapitre 1} présente l’étude du système actuelle : le contexte, la problématique, les besoins, et la méthode de travail et exploration des  solutions similaires;
  \item Le \textbf{chapitre 2} décrit l’analyse et la conception fonctionnelle et technique du système ;
  \item Le \textbf{chapitre 3} est consacré à l’implémentation, aux tests et à l’évaluation de la solution ;
  \item Enfin, la \textbf{conclusion générale} résume les apports du projet et propose des perspectives d’amélioration.
\end{itemize}
