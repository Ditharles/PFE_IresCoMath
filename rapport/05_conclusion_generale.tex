\chapter*{Conclusion générale}
\addcontentsline{toc}{chapter}{Conclusion générale}

Ce projet de fin d’études s’est inscrit dans le cadre de notre formation en Licence Génie Logiciel à la Faculté des Sciences de Gabès. Il a été réalisé au sein du laboratoire IRESCOMATH, et a porté sur la conception et le développement d’une plateforme web de gestion des demandes administratives et de matériels.

Dans un contexte universitaire où les demandes sont souvent traitées manuellement, cette solution vise à centraliser et automatiser le processus, en intégrant les rôles des différents acteurs (étudiant-master, doctorant, enseignant-chercheur, directeur). La plateforme offre des fonctionnalités essentielles telles que la création et le suivi de demandes, la validation hiérarchique, l’archivage ainsi que la notification en temps réel.

Sur le plan technique, nous avons utilisé des technologies modernes : React, TypeScript, Tailwind CSS, Node.js, PostgreSQL, et Uploadthing, avec une architecture client-serveur bien structurée. Le développement s’est appuyé sur la méthode agile Scrum, ce qui nous a permis de progresser par itérations avec des tests continus.

La réalisation de ce projet nous a permis de renforcer nos compétences en ingénierie logicielle, de travailler en binôme de manière efficace et de mettre en pratique les acquis théoriques de notre formation. Elle nous a également sensibilisés à l’importance de la gestion des besoins réels dans un cadre institutionnel. Cela nous pousse naturellement à détailler, dans la suite, les perspectives d’amélioration envisagées.

\bigskip

\textbf{Perspectives d’amélioration :}
\begin{itemize}
  \item \textbf{Renforcement de la sécurité} : Amélioration du système d'authentification avec l'implémentation de l'authentification multi-facteurs (2FA) et la gestion avancée des sessions utilisateurs pour garantir un accès sécurisé aux données sensibles ;
  
  \item \textbf{Validation électronique} : Intégration de signatures électroniques conformes aux standards juridiques pour renforcer la validation des documents administratifs et assurer leur authenticité ;
  
  \item \textbf{Analytics et reporting avancés} : Développement de filtres multicritères intelligents et implémentation d'un système d'export automatisé vers Excel avec génération de tableaux de bord dynamiques pour faciliter l'analyse et la prise de décision ;
  
  \item \textbf{Optimisation de la gestion matérielle} : Mise en place d'un système de traçabilité complet avec codes QR/RFID pour le suivi en temps réel des équipements, automatisation des processus de retour après prêt et amélioration de l'inventaire ;
  
  \item \textbf{Intelligence artificielle intégrée} : Déploiement d'agents IA pour l'automatisation intelligente des communications (emails personnalisés, notifications proactives) et l'optimisation dynamique des formulaires et templates selon les contextes d'utilisation.
 \end{itemize}

